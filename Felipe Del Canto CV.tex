% Remember to Typeset with XeLaTeX

\documentclass[a4paper,10pt]{article}

%A Few Useful Packages
%\usepackage{array}
\usepackage{etoolbox}
\usepackage{marvosym}
\usepackage{fontspec}
\usepackage{xunicode,xltxtra,url,parskip} 
\RequirePackage{color,graphicx}
\usepackage[usenames,dvipsnames]{xcolor}
\usepackage[big]{layaureo}
\usepackage{supertabular}
\usepackage{titlesec}
\usepackage{polyglossia}
\setdefaultlanguage{spanish}
\usepackage{datetime}

%Setup hyperref package, and colours for links
\usepackage{hyperref}
\definecolor{linkcolour}{rgb}{0,0.2,0.6}
\hypersetup{colorlinks,breaklinks,urlcolor=linkcolour, linkcolor=linkcolour}

%Spanish month names
\newcommand\Monthname[1][EMPTY]{%
  \ifnum1=#1Enero\else
  \ifnum2=#1Febrero\else
  \ifnum3=#1Marzo\else
  \ifnum4=#1Abril\else
  \ifnum5=#1Mayo\else
  \ifnum6=#1Junio\else
  \ifnum7=#1Julio\else
  \ifnum8=#1Agosto\else
  \ifnum9=#1Septiembre\else
  \ifnum10=#1Octubre\else
  \ifnum11=#1Noviembre\else
  \ifnum12=#1Diciembre\else
  \fi\fi\fi\fi\fi\fi\fi\fi\fi\fi\fi\fi
}

% Tab command
\newcommand{\tablength}{}
\newcommand{\setTabParams}[1]{\renewcommand\tablength{}\forcsvlist{\listadd\tablength}{#1}}

\newcommand{\setCols}[1]{			%
	\ifnum0=\i						%
		\ifdim0cm=#1				%
			\def \firstCol {r}		%
		\else						%
			\def \firstCol {p{#1}}		%
		\fi						%
	\else \ifnum1=\i				%
		\ifdim0cm=#1				%
			\def \secondCol {l}		%
		\else						%
			\def \secondCol{p{#1}}	%
		\fi						%
	\else \ifnum2=\i				%
		\ifnum0=#1				%
			\def \sep {}			%
		\else						%
			\def \sep {|}			%
		\fi						%
	\fi \fi \fi						%
	\advance\i by1					%
}

\newcommand{\tab}[1]{					%
	\newcount\i						%
	\forlistloop{\setCols}{\tablength}		%
	\begin{tabular}{\firstCol \sep \secondCol}	%
		#1							%
	\end{tabular} \\						%
}

%FONTS
\defaultfontfeatures{Mapping=tex-text} 
%\setmainfont[SmallCapsFont = Fontin SmallCaps]{Fontin}
%%% modified for Karol Kozioł for ShareLaTeX use
\setmainfont[
SmallCapsFont = Fontin-SmallCaps.otf,
BoldFont = Fontin-Bold.otf,
ItalicFont = Fontin-Italic.otf
]
{Fontin.otf}
%%%

\titleformat{\section}{\Large\scshape\raggedright}{}{0em}{}[\titlerule]
\titlespacing{\section}{0pt}{10pt}{7pt}

%--------------------BEGIN DOCUMENT----------------------
\begin{document}

\pagestyle{empty} % non-numbered pages

\font\fb=''[cmr10]'' %for use with \LaTeX command

%--------------------TITLE-------------
\par{\centering
		{{\Huge Felipe \textsc{Del Canto}}	\\
		 {\large {\Monthname[\the\month]} de \the\year }
	}\par}

%--------------------SECTIONS-----------------------------------
%Section: Personal Data
\section{Información personal}
\setTabParams{0cm,0cm,0}

\tab{
    \textsc{Lugar y fecha de nacimiento:}	&	Santiago, Chile  | 9 de Agosto de 1992 						\\
    \textsc{Dirección:}   				&	Lynch Norte 388 Casa E, La Reina, Santiago, Chile 				\\
    \textsc{Teléfono:}	   				&	+56 9 6699 9859 										\\
    \textsc{email:}     					&	\href{mailto:felipe.delcanto@icloud.com}{felipe.delcanto@icloud.com},
     									\href{mailto:fndelcanto@uc.cl}{fndelcanto@uc.cl}
}

%Section: Education
\section{Educación}
\setTabParams{0cm,0cm,0}

\tab{
\textsc{2019}
	& Magíster en Economía, \textbf{Pontificia Universidad Católica de Chile (PUC-Chile)},			\\
	& Santiago, Chile															\\
	& \emph{Aggregate problems require aggregate solutions? When heterogeneity is expendable}	\\
}

\tab{	
\textsc{2017}
	& Licenciatura en Matemática, \textbf{Pontificia Universidad Católica de Chile (PUC-Chile)},\\
	& Santiago, Chile\\	
}

%Section: Docencia
\section{Experiencia Docente}
\setTabParams{0cm,12cm,1}

\tab{
\textsc{Ago - Dic}	
		&	Ayudante, Microeconomía II (EAE211B), PUC-Chile \\
\textsc{2019}
		&	\emph{Docencia y corrección}	\\
		&	\footnotesize{Curso para estudiantes de Ingeniería Comercial. Profesor: Felipe Zurita.}	\\
}

\tab{
\textsc{Ago - Dic}	
	&	Ayudante, Economía Matemática (EAE319B), PUC-Chile\\
\textsc{2018}
	&	\emph{Docencia y corrección}	\\
	&	\footnotesize{Curso para estudiantes de Magister en Economía e Ingeniería Comercial, mención Economía. Profesor: Jaime Casassus.}	\\
}

\tab{	
\textsc{Ago - Dic}
 		&	Ayudante, Cálculo III (MAT1136), PUC-Chile \\
\textsc{2017}
		&	\emph{Docencia y corrección}	\\
		&	\footnotesize{Curso para estudiantes de Licenciatura en Matemáticas. Profesor: Martin Chuaqui.}	\\
}

\tab{	
\textsc{Ago - Dic}
 	&	Ayudante, Introducción a la Economía (EAE105A), PUC-Chile \\
\textsc{2017}
	&	\emph{Docencia y corrección}	\\
	&	\footnotesize{Curso para estudiantes de Geografía. Profesor: Pinjas Albagli.}	\\
}

\tab{
\textsc{Mar - Jul}
 	&	Ayudante, Álgebra Lineal (MAT1226), PUC-Chile \\
\textsc{2017}
	&	\emph{Docencia y corrección}	\\
	&	\footnotesize{Curso para estudiantes de Licenciatura en Matemáticas. Profesor: Jan Kiwi.}	\\
}

\tab{
\textsc{Ago - Dic}
 	&	Ayudante, Álgebra Lineal (MAT1226), PUC-Chile \\
\textsc{2016}
	&	\emph{Docencia y corrección}	\\
	&	\footnotesize{Curso para estudiantes de Licenciatura en Matemáticas. Profesor: Alejandro Ramirez.}	\\
}

\tab{
\textsc{Mar - Jul}
 	&	Ayudante, Introducción al Cálculo (MAT1106), PUC-Chile. \\
\textsc{2016}
	&	\emph{Docencia y corrección}	\\
	&	\footnotesize{Curso para estudiantes de Licenciatura en Matemáticas. Profesor: Duvan Henao.}	\\
}

\tab{
\textsc{Ago - Dic}
 	&	Ayudante, Cálculo I (MAT1506), PUC-Chile \\
\textsc{2015}
	&	\emph{Docencia y corrección}	\\
	&	\footnotesize{Curso para estudiantes de Agronomía. Profesor: Álvaro Cofré.}	\\
}

\tab{
\textsc{Ago - Dic}
 	&	Ayudante, Cálculo I (MAT1610), PUC-Chile \\
\textsc{2015}
	&	\emph{Docencia y corrección}	\\
	&	\footnotesize{Curso para estudiantes de Ingeniería. Profesor: Luis Dissett.}	\\
}

\tab{
\textsc{Mar - Jul}
 	&	Ayudante, Introducción al Cálculo (MAT1600), PUC-Chile \\
\textsc{2015}
	&	\emph{Docencia y corrección}	\\
	&	\footnotesize{Curso para estudiantes de College de Ciencias Naturales y Matemáticas. Profesor: Mario Ponce.}	\\
}

\tab{
\textsc{2014 - 2015}
 	&	Jefe de Departamento de Matemática, Preuniversitario Gauss \\
	&	\footnotesize{Encargado de la planificación de los cursos de Matemáticas del Preuniversitario, de las jornadas extra programáticas del área, de la calidad y renovación del material didáctico utilizado y del desempeño de los profesores.}	\\
}

\tab{
\textsc{2013 - 2015}
 	&	Profesor, Preuniversitario Gauss, Santiago, Chile \\
	&	\emph{Profesor de Matemática y Química}	\\
}

%Section: Work History
\section{Experiencia Laboral}
\setTabParams{0cm,11cm,1}

\tab{
\textsc{2017 - 2018}
 	&	\textbf{FONDECYT 1170178. Investment Strategies And Risk-Control Methods For Pension Funds: An Optimization-Based Approach.}	\\
	&	Asistente de investigación \\
}

%Sección: Publicaciones
\section{Publicaciones}
\setTabParams{0cm,11cm,1}

\tab{
\emph{Enviado}
 	&	\textbf{The effect of regularization in portfolio selection problems} \\
\textsc{2019}
	&	\footnotesize{B.K. Pagnoncelli, F. Del Canto, A. Cifuentes}	\\
}
	
%Section: Presentaciones
\section{Presentaciones}
\setTabParams{0cm,11cm,1}

\tab{
\textsc{Octubre}
 	&	\textbf{3rd Eastern Conference on Mathematical Finance, Illinois Institute of}	\\
\textsc{2018}
	&	\textbf{Technology, Illinois, United States} \\
	&	The effect of regularization in portfolio selection problems	\\
	&	\emph{Presentación de póster}
}

%Section: Scholarships and additional info
\section{Becas}
\setTabParams{0cm,0cm,0}

\tab{
\textsc{2018-2019} 
	&	Beca de Magíster Nacional, Programa Formación de Capital Humano Avanzado	\\
	&	CONICYT, Chile	\\
	&	\\

\textsc{2015-2017} 
	&	Beca Rolando Chuaqui, Facultad de Matemáticas	\\
	&	PUC-Chile	\\
	&	\\

\textsc{2015}
	&	Matrícula de Honor (Beca)	\\
	&	PUC-Chile	\\
	&	\\
}

%Section: Languages
\section{Idiomas}
\setTabParams{0cm,0cm,0}

\tab{
 \textsc{Español:}
 	&	Nativo	\\
\textsc{Inglés:}
	&	Fluido	\\
}


% Section: Computer Skills
\section{Competencias computacionales}
\setTabParams{0cm,0cm,0}

\tab{
Conocimiento básico:
	& Excel, Word, PowerPoint, Prezi, Java, Swift, Unity 	\\

Conocimiento avanzado:
	& {\fb \LaTeX}, Python, Stata, MATLAB, Gurobi, C\#	\\
}


%\newpage
%\par{\centering\Large \hypertarget{grds}{Master of Science in \textsc{Finance}}\par}\large{\centering Grades\par}\normalsize
%\begin{center}
%\begin{tabular}{lcc}
%\multicolumn{1}{c}{\textsc{Exam}}&\textsc{Grade}&\textsc{Credit Hrs}\\ \hline
%Corporate Finance (Valuation)	&25&	6\\
%Financial Statement Analysis	&28&	6\\
%Statistics	&27&	6\\
%Theory of Finance	&26&	6\\
%Quantitative Methods for Finance	&30&	6\\
%Econometrics	&24	&6\\
%Derivatives	&31&	6\\
%Management of Financial and Insurance Companies	&30&	6\\
%Business Law	&31&	6\\
%Investment Banking	&28&	6\\ \\
%		
%Behavioral Models for Economics and Finance	&29&	6\\
%Numerical Methods for Finance	&29&	6\\
%Advanced Derivatives	&30&	6\\
%Fixed Income (Advanced Methods)	&30&	6\\ \\
%		
%English Language	&30&	4\\
%French Language	&31&	4\\
%		
%Internship	&	&8\\
%		
%Final Thesis	&	&20\\
%		
%		& Total&120\\\cline{2-3}
%&\textsc{Gpa}&\textbf{28.61}
%\end{tabular}
%\end{center}
%\bigskip
%\hrule
%\bigskip
%\par{\centering\Large \hypertarget{grds_cleli}{Undergraduate Degree in \textsc{Law and Business Administration}}\par}\large{\centering Grades\par}\normalsize
%
%\begin{center}
%
%\tablefirsthead{%
%  \multicolumn{1}{c}{\textsc{Esame (\textsc{ita})}}&\multicolumn{1}{c}{\textsc{Exam (\textsc{eng})}}&\textsc{Grade}&\textsc{Credit Hrs}\\ \hline}
%\tablehead{%
%\multicolumn{1}{c}{\textsc{Esame (\textsc{ita})}}&\multicolumn{1}{c}{\textsc{Exam (\textsc{eng})}}&\textsc{Grade}&\textsc{Credit Hrs}\\ \hline
%}
%\tabletail
%\tablelasttail{}	
%
% \begin{supertabular}{p{4.9cm} p{4.9cm} c c}
%Economia aziendale&Theory and principles of management& 28 & 8\\
%Istituzioni di diritto privato&Principles of private law&30&8\\
% Istituzioni di diritto pubblico 
%&
% Principles of public law 
%& 30&
% 6 
%\\
%Matematica generale 
%& Mathematics 
%&31&8\\
% Contabilit\`a e Bilancio 
%& Accounting and Financial statements 
%& 31 & 8\\
% Diritto del lavoro 
%&
% Labour law 
%& 30&
%4 
%\\
% Informatica & Computer skills &29& 4 
%\\
% Economia e gestione delle imprese & Corporate management &31&6 
%\\
% Microeconomia & Microeconomics &30&8 
%\\
% Contabilit\`a e bilancio 2 & Accounting and financial statements 2 
%&29&8 \\
% Diritto commerciale & Company and business law &30&8 
%\\
% Macroeconomia & Macroeconomics &29&8 
%\\
% Statistica & Statistics &30&6\\
% Diritto delle procedure concorsuali & Insolvency law &31&4 
%\\
% Finanza aziendale 
%& Corporate finance 
%&31 &8 \\
% Matematica finanziaria & Financial mathematics &31&4 
%\\
% Programmazione e controllo 
%& Management accounting &31&8 
%\\
%Lingua Inglese \textsc{c1}& English Language \textsc{c1}& 25&6\\ 
%Diritto tributario & Tax law &30&6 
%\\
%Organizzazione e sistemi informativi aziendali &Management information systems &28&6 
%\\
% Strategia e politica aziendale* 
%& Business strategy* & 29& 6 
%\\
%Derivati*&Derivatives*&30&6\\
%Opzionale all'estero* & Corporate Financial Strategy* &30&6\\
%Lingua Francese \textsc{b1} & French Language \textsc{b1}&31&6\\
%Economia dei mercati e degli intermediari finanziari* 
%&Financial markets and institutions* &30&6 
%\\
%Revisione aziendale 
%&Financial auditing &31&6 
%\\
%Scienza delle finanze &Public economics &30&6 
%\\
% Lavoro finale& Final report&&6\\
%& & Total&180\\\cline{3-4}
%& &\textsc{Gpa}&\textbf{29.85}\\ \\ \multicolumn{4}{l}{\footnotesize A star (*) indicates that the course was taken at the \textbf{University of Southern California}, Los Angeles, \textsc{usa}}
% \end{supertabular}
%\end{center}
%\bigskip
%\hrule
%\bigskip
%\par{\centering\Large \hypertarget{grds_usc}{Exchange Program at \textsc{usc}, Los Angeles}\par}\large{\centering Grades\par}\normalsize
%
%\begin{center}
%\begin{tabular}{lcc}
%\multicolumn{1}{c}{\textsc{Exam}}&\textsc{Grade}&\textsc{Grade Points}\\ \hline
%Corporate Financial Strategy	&A&	4\\
%Derivatives	&A&	4\\
%Money, Credit, and Banking	&A&	4\\
%Business Strategy & A-& 3.5\\
%& &\\\cline{2-3}
% &\textsc{Gpa}&\textbf{3.875}
%\end{tabular}
%\end{center}
%%\newpage
%%\hypertarget{gmat}{\textsc{Gmat}\setmainfont{LMRoman10 Regular}\textregistered\setmainfont[SmallCapsFont=Fontin-SmallCaps]{Fontin-Regular}}
%
%%\XeTeXpdffile ''GMAT.pdf'' page 1 scaled 800

\end{document}
